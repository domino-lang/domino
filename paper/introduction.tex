\section{Introduction}
Halevi's dream-vision of computer-aided reduction proofs in cryptography~\cite{Halevi} suggests that creativity is reserved for humans, while algorithms are in charge of tedious steps which confirm the soundness of a reduction and only use standard techniques. An important success on the path for computer-aided cryptography is that computer-aided \emph{attack-finding} has established itself firmly as a standard tool for protocol development. Most importantly, Tamarin~\cite{X} has been used to improve ...~\cite{X}, ...~\cite{X} and ...~\cite{X}, and, in particular, is used across the community also without involving an author of the original tool, see, e.g.~\cite{X}. However, computer-aided \emph{reduction proofs} have not yet joined the quest for support

==== OLD ====

However, invoking the help of a computer requires formalizing one's proofs at a level of detail which is unusual for standard mathematical practice, including cryptography. For example, one might describe a two security games and a construction in English, then describe a reduction in English and prove its soundness by arguing about difficult key points.

While computer-aided attack-finding has become a standard tool in protocol development


In a celebrated breakthrough, Abadi and Rogaway~\cite{X} showed that 









==== EVEN OLDER ====


Since Halevi~\cite{X} declared a \emph{crisis of rigour} for reduction proofs in cryptography
and called for tool support to leave the \emph{interesting} steps to the creative cryptographer
and the \emph{mundane} steps to a powerful automated tool, significant progress has been made
towards this vision. In particular, Dupressoir and some others~\cite{X} formalized the Bellare-Rogaway
code-based game-playing~\cite{X} approach in EasyCrypt which now allows to verify code transformations
and reduction-based game hops. 

EasyCrypt does not take care of mundane steps fully automatically, as in Halevi's vision, but rather is semi-automated
and usually requires cryptographers to write pre-conditions and post-conditions (or simply a single invariant) and prove
that if the pre-condition was satisfied before an oracle call, then the post-condition is satsified.
The reason that the SMT solver underlying EasyCrypt cannot prove code equivalence without additional
help is that conceptual high-level reasoning is not part of the solver. The present paper does not
change this state of affairs, but instead increases \emph{modularity} of the EasyCrypt technique.

Concretely, we present SSPVerif, a new verification tool for cryptographic game-hopping proofs which
is based on \emph{state-separating proofs} (SSPs), a recent, modular technique for game-hopping in pen-and-paper
proofs. SSPs structure code of cryptographic protocols and security game as stateful pieces of code
called \emph{packages} which can call one another, but otherwise do not share state. Since package
composition is associative, one can simplify complex game-hops by ``pushing code into the adversary''
(cf. Section~\ref{X}).

SSP-style techniques have already been formalized in EasyCrypt~\cite{X}, but since package are not
native in EasyCrypt, the implementation does not provide SSP-specific support and, in particular,
the SSP-style workflow. SSPs have also been formalized in SSProve~\cite{X}, a Coq-based tool specially
developed for SSPs, but no large examples have been shown in SSProve. In turn, SSPVerif has been developed
to especially support the workflow of the working SSP cryptographer.

As an example, we use Yao's garbled circuits: much larger than what is known in SSProve; Yao has already 
been done on EasyCrypt, so great test case.