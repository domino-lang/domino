\section{State-Separating Proofs}\label{sec:ssp}
In this section, we review state-separating proofs (SSPs)
as introduced by Brzuska, Delignat-Lavaud, Fournet, Kohbrok and
Kohlweiss (BDEFKK~\cite{X}), integrating refinements and
suggestions proposed in subsequent works relying on SSPs~\cite{blanket-cite,konrads-thesis,sabinesworks}. While reviewing core SSP concepts,
we also explain their encoding in SSBee.

\subsection{Games and Packages}
\paragraph{Games.}
Following Bellare and Rogaway~\cite{Bellare-Rogaway}, a cryptographic game is a set
of oracles operating on shared state.
\begin{definition}
A game $\M{G}$ consists of a set of oracles
which operate on a shared state.
\end{definition}
For example, indistinguishability under chosen plaintext attacks (IND-CPA)
can be formalized as indistinguishability between two games $\M{IND-CPA}^b$, 
each of which provides an encryption oracle $\O{ENC}$ to the adversary that
either encrypts the left or the right input message. The state of 
$\M{IND-CPA}^b$ is only the symmetric encryption key $k$. See Fig.~\ref{X}
for details. Note that search games such as unforgeability under chosen
plaintext attacks (UNF-CMA) of message authentication codes (MACS) and, in general,
any search game can also be encoded as distinguishing games. See Appendix~\ref{app:dist} for details. This is important because SSPs restrict their attention to \emph{distinguishing games}.

\paragraph{Packages.}
If we want to decompose a game into multiple pieces of code, those pieces
of code need to be able to call one another. Therefore, the native citizen
of SSPs is a \emph{package} which generalizes the notion of a game in that
a package not only exposes oracles, but can also \emph{call} oracles (of
other packages, not of itself).

\begin{definition}[Package]
A package $\M{M}$ provides a set of oracles $[\rightarrow \M{M}]$
which operate on a shared state and make calls to a set of oracles
$[\M{M}\rightarrow]$ which we call the \emph{dependencies} of $\M{M}$.
\end{definition}

We can then construct games by modularly describing them as a composition
of several packages.

\paragraph{SSBee.} In SSBee, the user defines each package by specifying
the parameters (e.g., security parameter, encryption schemes), state
variables (e.g., the key $k$ in the IND-CPA game) as well as pseudo-code
for the oracles. The content of variables other than state variables will
be erased in the end of an oracle call. See Fig.~\ref{X} for a description
of the IND-CPA package in SSBee.

\paragraph{Composed games.} In order to specify a (composed) game in SSBee, 
we need to specify the packages that it consists of as well as their wiring,
i.e., which package calls which oracles of which other package as well as
which oracles are exposed to the adversary. For example, let us
decompose the IND-CPA game into two packages, 
a $\M{KEY}$ package (cf. Fig.~\ref{X})
which stores the key $k$ and a stateless 
$\M{ENCRYPT}^b$ package (cf. Fig.~\ref{X}). Now, the composition ...
specifies that $\M{ENCRYPT}^b$ calls the $\O{GET}$ oracle of $\M{KEY}$,
and the adversary can call the $\O{SAMPLE}$ oracle of $\M{KEY}$ and the
$\O{ENC}$ oracle of $\M{ENCRYPT}^b$.

{\color{blue}Chris: Can compositions be parametrized with a bit $b$ and
use different packages based on which value $b$ has?}

\subsection{Proofs}
standard pattern and how we use this for SSBee. Code equivalence proofs.

\paragraph{SSBee.}

\paragraph{Hybrid arguments.}
